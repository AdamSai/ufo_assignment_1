\documentclass{report}
\title{Bachelor Thesis Template}
\author{Adam Saidane}
\date{October 2020}
\usepackage{graphicx}
\usepackage[section]{placeins}
\usepackage{multirow}
\graphicspath{ {./img/} }
\usepackage[nottoc]{tocbibind}
\usepackage{listings}
\usepackage{color}
\usepackage{todonotes}
\lstloadlanguages{C,C++,csh,Java}

\definecolor{red}{rgb}{0.6,0,0} 
\definecolor{blue}{rgb}{0,0,0.6}
\definecolor{green}{rgb}{0,0.8,0}
\definecolor{cyan}{rgb}{0.0,0.6,0.6}
\definecolor{cloudwhite}{rgb}{0.9412, 0.9608, 0.8471}

\lstset{
language=csh,
basicstyle=\footnotesize\ttfamily,
numbers=left,
numberstyle=\tiny,
numbersep=5pt,
tabsize=2,
extendedchars=true,
breaklines=true,
frame=b,
stringstyle=\color{blue}\ttfamily,
showspaces=false,
showtabs=false,
xleftmargin=17pt,
framexleftmargin=17pt,
framexrightmargin=5pt,
framexbottommargin=4pt,
commentstyle=\color{green},
morecomment=[l]{//}, %use comment-line-style!
morecomment=[s]{/*}{*/}, %for multiline comments
showstringspaces=false,
morekeywords={ abstract, event, new, struct,
as, explicit, null, switch,
base, extern, object, this,
bool, false, operator, throw,
break, finally, out, true,
byte, fixed, override, try,
case, float, params, typeof,
catch, for, private, uint,
char, foreach, protected, ulong,
checked, goto, public, unchecked,
class, if, readonly, unsafe,
const, implicit, ref, ushort,
continue, in, return, using,
decimal, int, sbyte, virtual,
default, interface, sealed, volatile,
delegate, internal, short, void,
do, is, sizeof, while,
double, lock, stackalloc,
else, long, static,
enum, namespace, string},
keywordstyle=\color{cyan},
identifierstyle=\color{red},
backgroundcolor=\color{cloudwhite},
}

\usepackage{caption}
\DeclareCaptionFont{white}{\color{white}}
\DeclareCaptionFormat{listing}{\colorbox{blue}{\parbox{\textwidth}{\hspace{15pt}#1#2#3}}}
\captionsetup[lstlisting]{format=listing,labelfont=white,textfont=white, singlelinecheck=false, margin=0pt, font={bf,footnotesize}}



\begin{document}
\begin{titlepage}
   \begin{center}
       \vspace*{1cm}

       \textbf{Bachelor Thesis Template}

       \vspace{0.5cm}
        Thesis Subtitle
            
       \vspace{1.5cm}

       \textbf{Adam Saidane}

       \vfill
            
       A thesis presented for the degree of\\
       Bachelor of Software Development
            
       \vspace{0.8cm}
     
       \includegraphics[width=0.4\textwidth]{cphbusiness_logo}
            
       Lyngby\\
       Copenhagen Business Academy\\
       Denmark\\
       October 2020
            
   \end{center}
\end{titlepage}
\tableofcontents
\listoftodos
\section{Graphics}

\begin{figure}[ht]
	\caption{Mankrik, a  man who has lost his wife in the war with the quilboars}
	\includegraphics[width=0.5\textwidth]{mankrik}
	\label{mankrik}
\end{figure}

\begin{figure}[!ht]
	\centering
	\includegraphics[width=0.5\textwidth]{quilboar}
	\caption{The assailant who murdered the wife of the man from figure \ref{mankrik} on page \pageref{mankrik}}

\end{figure}





\begin{figure}[!ht]
\centering
\begin{minipage}{0.4\textwidth}
	\includegraphics[width=\textwidth]{mankrik_young}
	\caption{A young Mankrik}
\end{minipage}
\begin{minipage}{0.4\textwidth}
	\includegraphics[width=\textwidth]{mankrik_wife}
	\caption{Mankrik's wife}
\end{minipage}

\end{figure}
\missingfigure{insert a picture of Arthas}
\todo{add a description to the missing picture}
\FloatBarrier
\section{Sections}
This is a section
\subsection{Subsection}
This is a subsection
\subsubsection{Sub sub section}
This is a subsubsection :)
\paragraph{This could be}
 a longer paragraph where we go into detail about a specific issue. This could be any issue at all, which you want to write about.
 \subparagraph{A subparagraph} could be used when you need to go into more detail about an issue you're writing about in the paragraph.
\section*{Lists (This is a non-numbered section)}
\begin{itemize}
	\item first item in the list
	\item second item in the list
\end{itemize}

\begin{enumerate}
	\item first item in the list
	\item second item in the list
\end{enumerate}
\renewcommand{\labelenumi}{\Roman{enumi}}
\begin{enumerate}
	\item first item
	\item second item
\end{enumerate}

\renewcommand{\labelitemi}{$\textasteriskcentered$}
\begin{itemize}
 	\item first item
 	\item second item
\end{itemize}
\section{Table with multiple columns}


\begin{center}
\begin{table}[h]
\caption{\label{tab: Table showoff} showing off my tables}
\begin{tabular}{ |l|c|r| } 
 \hline
 cell one left aligned & cell two center aligned & cell three right aligned \\ 
 \hline
 cell four l aligned & cell five c aligned & cell six r aligned \\ 
 \hline
 \multicolumn{3}{|c|}{this spans multiple columns} \\
 \hline
 \multirow{2}{*}{Multirow}&X & \\
 &X &\\
 \hline
\end{tabular}
\medskip

Here is a description, describing that this table above, shows you a few things you can do with tables
\end{table}
\end{center}
Please view table \ref{tab: Table showoff} to get an idea of what you can do with tables.
\section{Code listing}

\begin{lstlisting}[language={[Sharp]C}, caption={C\# exaple}, label={Script}]
public static void main(string[] args) {
	Console.WriteLine("Hello World");
	var myVar = 10;
	var hisVar = 20;
	var max = myVar >= hisVar ? myVar : hisVar;
}
\end{lstlisting}
\todo{change colors of the coding style \ldots}
\section{Citations}
Here is a citation from an Einstein article\cite{einstein}. And here is a quote from a book\cite{latexcompanion}. Lastly here is a quote from a website\cite{knuthwebsite}

	\bibliographystyle{plain}
	\bibliography{references}

\end{document}
